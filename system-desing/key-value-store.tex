\chapter{Design a Key Value Store}

\ha{What is a Key Value Store?}
A key value store in a distriburted system is a module that works as a distributed map.

\ha{Use Cases}
Distributed key-value shows how basics thing like CAP, replication, partiotioning,fault-torence, conflict resolution works in a distributed system.

\ha{Requirements}
\lstart
    \i key-value pair of size 10KB
    \i High availiblity: The system should response quickly, even during failure.
    \i High scalibity: The system should be scalable to support highter load in case the need arise.
    \i Server Addition and removal should take less time and consue less resources.
    \i tunable consistency via \u{quoram consenses}
\lend

\ha{Desing}

\hb{Single Server Store}
Discuss data compression to save space. cache for most used key and rest in disk.

Discuss why it cannot be used in distributed system and how scaling it is a challenging task.

\hb{Distributed Server Store}

Flow:
\lstart
    \i Discuss CAP theorem.
    \i Why partitions cannot be avoided in distributed system?
    
    \todo{Add fig of 3 system}
    If we chose consistency over availiblity(CP system),then we must block all write operation to n1 and n2 to avoid data inconsistency. (data inconsitecy amoungh these server can make the system unavailable)

    If we choose availability over consistency(AP System), then the system keep accepting reads,even though it might return stale data.

    Chosing the right CAP guarntees that the system will work according to designer requirements.

    Using quorum consensus we can fine tune consistency level.

    \vspace{0.7cm}
    Now you can continue discussing below system component.

    \i Data Partition 
    \i Data replication
    \i Consistency tuning | quorum consenses | consistency models
    \i Inconsistency Resolution | vector clock
    \i Handling failure | gossip protocol | sloppy quoram + Hinted Handoff | Merkle tree + anti-entryopy protocol 
    
    \vspace{0.5cm}
    \i system architecture diagram
    \i write path
    \i read path


\lend

\todo{add figure6-18}

\todo{add fig6-20, fig6-21}
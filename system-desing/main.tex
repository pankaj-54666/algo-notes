\documentclass{../latex-setting/cmemoir}
\usepackage{scrextend}
\usepackage{minitoc}
\usepackage{shortvrb}

% Question-Anser Summary Commands
\newcounter{qcount}


\newenvironment{test}[1]
{
    \textbf{#1}
}
{}

\newenvironment{question}
{
    \stepcounter{qcount}
    \textbf{(\theqcount)}
}
{}

\newenvironment{answer}
{
    \begin{addmargin}[10em]{0em}
    \textbf{Answer:}

}
{\end{addmargin}}
\newcommand{\qs}{%
    % \stepcounter{qcount}%
    \begin{question}%
}
\newcommand{\qe}{ 
    \end{question}
}

\newcommand{\as}{%
    % \stepcounter{qcount}%
    \begin{answer}%
}
\newcommand{\aend}{ 
    \end{answer}
}


\newenvironment{qsummary}
{
    \newcommand{\qstart}
}
{}

\begin{document}

\chapter{Web Crawler}


% \qus{
%     What is a web crawler? 
    
%     \includegraphics[width=\marginparwidth]{example-image-a}
% }

% \ans{

% }

\qs What is a web crawler? \verb|qstart| \qe

\qs What is a web crawler? 
    
    \includegraphics[width=\marginparwidth]{example-image-a} \qe

\as
    A web crawler is a system (made of multiple modules) that explores the internet for specific purpose.
\aend

\begin{question} 
    Question2 \marginnote{Hello marginnote}
\end{question}

\begin{compactenum}
    \item question1
    \item Question2
    \item question3
    
\end{compactenum}



\begin{test}{What is a web crawler?}
   
    A web crawerl is a system that explores the word wide web (internet) to carry out specific task.

    Some example of the web crawerl is.
\end{test}

\end{document}

\begin{comment}
    \question{

    }
    \solution{

    }

    Q:
    A:

    Q;
    A:


\end{comment}
\documentclass{../latex-setting/cmemoir}
\usepackage{scrextend}
\usepackage{minitoc}
\usepackage{shortvrb}
\usepackage{system-design} %Style For system-desing book


\begin{document}
\tableofcontents


\chapter{Web Crawler}


\begin{exercise}[Web Crawler Design Question/Summary:]
\begin{enumerate}
    \item What is a web crawler system? 
    \item Given some example in which internet is explored for specific purpose.
    \item How many pages will the crawler will process  approximalely? (reply: 1B per month)
    \item What do we need to store?
    
\end{enumerate}

\end{exercise}

\qicon What do we need to store?
\sicon HTML only for now.

\qicon How long we need to store the crawl result?
\sicon Lets assume we need to store 5year worth of data.



\newpage
\label{answers}
\begin{question}{What is a web crawler?}
   
    A web crawerl is a system that explores the word wide web (internet) to carry out specific task. Some of the examples what normall web-crawlers are used are:
    \begin{compactenum}[(i)]
        \item Index Generation for search gian like google,yahoo etc.
        \item Taking a time-snapshot of world wide web for archival or library purpose.
        \item Detect copyright infrignment.
        \item Detect sensitive code re-usage by anyone on the web.
        \item Extract specific images for modal training.
        \item etc \dots
    \end{compactenum}

\end{question}

\qq { What are the systems that gets involved in desinging a large application?}



\chapter{Web Crawler}

\qs{What is a web crawler?}
    A web crawler is a system that explores the internet with specific goal in mind. The goal can be be something like. (aka applicatio of web crawler)
    \ls
        \i Generate a snapshot of Internet at current time.
        \i Browse the internet for code duplicacy / piracy.
        \i Browse the web for search index generation for gians like Google Search.
        \i  Copyright Violation finding on document shared on internet.
        \i Research finding of something related to some search.
        \i download  all humans-dog interaction pics from instagram for ML modal training with the captions for machine modal traning.
    \le
\qe

\q{List Some Applications of web crawler.}


\ta{Design Considerations}
Now as we know what the system does, lets finilize what all feature we are required to implement from this system. Also, with what performace (latency, availibity, consistency) is expected from our system.

\lstart
    \i How many page download per month are we expecting? 
    \r{let's suppose it 1B per month.}
    
    \i Are we desiging for a single website (like school webcrawler to see suspicision activtiy from each user profile) or we are planning for whole web?
    \r{We are planning for whole web.}

    \i What is the document we are designing our crawler for?
    \r{For now lets just say we need HTML, but our system should be flexible to extend for other type also with minimum changes.}


\lend
\include{cap.tex}


% \verba{txt/notification-system.txt}

% \verbatiminput{txt/notification-system.txt}
\lstinputlisting{txt/notification-system.txt}
% TO-DO: include img files also


\chapter{News Feed System}

\ha{What is News Feed Systesm?}

A news feed is sytems that aggregate news/post/code other informatin from a set of post by other user/system. It thens shows them in desired order to the user. The system target is to provide information to the system user. One of the indirect goal of the system in to suggest content to user so that they will spend much of thier time on the web.

\hb{What the User can Post?}
User can post video,text,photo and other short info.


\hb{Real Life Use Case}
Some of the examples of system use are:
\ls
    \i Facebook News Feed System
    \i Instrgram Feed
    \i Google News Feed (data is generated from web itself)
    \i Twitter Timeline
\le

\ha{Design Scope Aggrement}
We will be asking question to decide how complex our system need to be. Also we will clarify any question regarding the system and performace and user expectatino.

\ha{Functional Features}
\lstart
    \i What the user is allowed to post?
    \r{user can post video,photo,text and external links}

    \i Can other user like,dislike ,post and comment on the post?
    \r{Yes, but its secondard to your explanation as main features will be post and timeline.}

    \i DAU? 
    \r{Lets suppose its 1B}

    \i Do we need storage?
    \r Yes, lets suppose we need storeage for 5 years.

    \i Are we designing mobile only app? web app? or both?
    \r Its both, giving user the flexibility to use the system at their convience.

\lend

\hb{Non Functional Features}
\lstart
    \i Its okay if uploading of a video takes a little time.
    \i Timeline generation should not take long,as it will user experience and user may leave the app.
    \i post of celebrity need to take special care.
    \i generation of timelint at runtime is expensigve, so we need to pre-compute it on our desing.
\lend

\ha{HLD Proposal}
Once we have basic understanding of the system, we can now propose initial desing. This way we can be sure that we and the interviewer agree on what is expected from the system.

There are two main featuers of our system (a) Post Services (b) Timeline Generation.

% TO-DO: Decrease pic resolution for fast compilation
    \b{HLD Diagram}\\
    \includegraphics[width=0.7\textwidth]{resources/news-feed-hld.jpg}    



For posting, we will have our clients at either mobile/web. They will be post via exposed REST postAPI. They will be call the API and post the content. Once the webserver receives it forward the postrequest to postServices. It then save the data to DB, which is backed by cache.

For timeline generation, once the user opens the app then he will query the server. The server will check their friend list and their post,like etc and generate a JSON timeline response and send it to user.


\ha{HLD Evolution and LLD}
As we can observer there are various thing that make user experiance bad, and we can imporve them. Some of them are:
\lstart
    \i News Feed generation at run-time will take too much time. 
    \r We proposed alternative solution, in which news feed will be precomputed to save loading time on user time. Once any user post anything, then we pass that info to a new module called \u{Fanout Service}. The fanout service will get the user post and their friend list, and generate a (postID,userID) pair and push them into a queue. A set of worker thread keep reading this queue, it extracts a pair from the queue.And update userID timeline with consideration for the newly post. (if it want extra data it can query corresponding DB to get postDetails or userDetaisl.)

    \i With above approach, if the user has many friends / user the follow them. That that will lead to a heavy-increase in queue size, overall slowing the system. For these type of user, we will follow \u{Fanout on Read} instead of \u{Fanout On Write}. Once the user open their timeline, then recent post are pulled when user loads her homepage.

    \i For less-active user(who follows a celebrity), fanout on write works best. As computing resources are not waisted.

    \i How can we get list of friends?
    \r Instead of using SQL/No-SQl db, we use will be using \b{graphDB}  which works better with relationship data.

    \i Discuss userCache, postCache and newsFeedCache.
    \r These cases will be stored in front of DB, and instead of storing whole db-row, we will save metadata on the cache. (As original post is needed only if user loads the homepage, in all precomuption steps only their ID are sufficent.)

    \i Do we need CDN?
    \r Yes, as we know from socal media user want to see popular content. So if we place CDN also, static data load(like mangalyaan-3 landing) can be loaded fast.

    \i As we know we will need to be shard the userDB. Discuss the partiotioning strategy for best performace of the userDB and postDB.
\lend

\textbf{News Feed LLD}\\
\includegraphics[width=\textwidth]{resources/drawio/news-feed-lld.png}

\ha{Further Discussion Point}
As every system is complex, there will always be imporvement points. Some of the points can be applied to any system.
If time permits you can discuss and evolve the desing around these points alos:\

\hb{Scaling the DB}
\ls
    \i Vertical Scaling vs Horizontal Scaling
    \i SQL  vs No-SQL vs graphDB
    \i Master-Slave Replication and other replication strategy
    \i Read Replice and their use
    \i Consistely Models
    \i Databsse Sharding and avoiding common db problem like: hotspot problem
\le

\hb{Scaling the Desing}
\ls
    \i Keeping web tier stateless
    \i Caching and cache evicatioh policy and types of cache
    \i Use of multiple data center
    \i loose couple component with mesage queues
    \i Monitoring Features. (ex: QPS during peak hours, maximumm queue size etc)
    \i use of Notifcation system when post is generated.
\le


\chapter{Rate Limiter}

\ha{What is Rate Limter?}
A rate limiter is s system that limits the number of request that can be made to a system during a fixed time interval. It is used to controll incoming traffic or outgoing traffic. Server implement rate limiter to avoid overload of their system and avoid DDoS attack and better user experiance for majority of users.

In HTTP world, a rate limiter limits the number of client request allowed to be sent over a specified period. If the API request count exceeds the threshold defined by the rate limiter, all the excess calls are blocked.

\hb{Real World use case of rate limiting system.}
\lstart
    \i A user can write no more than 2 post per second.
    \i You can create a maximum of 10 accounts per day from same IP address.
    \i You can claim reward no more than 5 times per week from same device.
    \i Prevent resource starvation causes by DDoS attack.
    \i Reduce cost and cost approximation by reducing allowed request per minute.
    \i Allocating different API different priority  by assigining uneven timing to them.
\lend



\ha{Design Scope Discussion}

List of query that need to be finilized before starting the design.

\lstart
    \i It's a client side rate limiting or server side?
    \r Server Side.

    \i What are the parameter upon which the rate limiter will throttle the request? Is it IP address, userID or other properties?
    \r Let's say we want it to flexible.

    \i Number of calls to the Rate limiter?
    \r Lets say its 50M calls to the server upon which we want to intergrate the rate limiter.

    \i Is the system distriburted?
    \r Of Course.

    \i Do we need to inform the user that their request are beign throttled/dropped?
    \r Yes, a good system must not remain unambigous. We need to return 429 HTTP code in case the request get blocked due to rate limiting throttling. 
\lend

\hb{Non Funcitonal Requerement }
Here are some of the non functional requirement that we need to consider while desiging our system.

\lstart
    \i The system must not introduce significient latency. i.e it should be a \u{low latency} system, and should not slow down the HTTP request.
    
    \i Use minimum memory.
    \i  Exception Handling: return appropriate error-code upon different state.

    \i High Fault tolerance: If there is any problem with the rate limiter(ex: a cache servers goes offline), it should not affect the entire system.
\lend

\ha{Algorithms For Rate Limiting}

Where should the system be places?\\
It can be placed at API gateway or standalone version in between client and API server or can be buit inside of API server.

Rate Limting Algorithms.
\ls
    \i Token Bucket Algorithm.
    \i Leaking Bucket
    \i Fixed Widnow Counter
    \i Sliding Window long
    \i Sliding Window Counter
\le



\ha{HLD Proposal}
Now as we understand the basic idea of rate limiting. It need to maintains a counter variable, and if the counter is above threshold then the subsequent request will be dropped.


\ha{LLD | Desing Deep Dive}



\ha{Wrap Up}
\chapter{Design a Key Value Store}

\ha{What is a Key Value Store?}
A key value store in a distriburted system is a module that works as a distributed map.

\ha{Use Cases}
Distributed key-value shows how basics thing like CAP, replication, partiotioning,fault-torence, conflict resolution works in a distributed system.

\ha{Requirements}
\lstart
    \i key-value pair of size 10KB
    \i High availiblity: The system should response quickly, even during failure.
    \i High scalibity: The system should be scalable to support highter load in case the need arise.
    \i Server Addition and removal should take less time and consue less resources.
    \i tunable consistency via \u{quoram consenses}
\lend

\ha{Desing}

\hb{Single Server Store}
Discuss data compression to save space. cache for most used key and rest in disk.

Discuss why it cannot be used in distributed system and how scaling it is a challenging task.

\hb{Distributed Server Store}

Flow:
\lstart
    \i Discuss CAP theorem.
    \i Why partitions cannot be avoided in distributed system?
    
    \todo{Add fig of 3 system}
    If we chose consistency over availiblity(CP system),then we must block all write operation to n1 and n2 to avoid data inconsistency. (data inconsitecy amoungh these server can make the system unavailable)

    If we choose availability over consistency(AP System), then the system keep accepting reads,even though it might return stale data.

    Chosing the right CAP guarntees that the system will work according to designer requirements.

    Using quorum consensus we can fine tune consistency level.

    \vspace{0.7cm}
    Now you can continue discussing below system component.

    \i Data Partition 
    \i Data replication
    \i Consistency tuning | quorum consenses | consistency models
    \i Inconsistency Resolution | vector clock
    \i Handling failure | gossip protocol | sloppy quoram + Hinted Handoff | Merkle tree + anti-entryopy protocol 
    
    \vspace{0.5cm}
    \i system architecture diagram
    \i write path
    \i read path


\lend

\todo{add figure6-18}

\todo{add fig6-20, fig6-21}

Resouces:
\lstart
    \i https://bytebytego.com/
    \i Alex Xu Book 1
    \i ALex Xu Book2
    \i Grokking the system desing?
\lend


\end{document}
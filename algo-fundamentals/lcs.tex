% Longest Common subsequeuen

\begin{problem}{Longest Common Subsequence(LCS)}
    Given two string s1 and s2. Find a new string which is the longest common substring of the given string.
    Practice: LC1143
\end{problem}

\begin{solution}[Inclusion-Exclusion]
    \obeylines
    In all string problem, we require to create some final string.
    In some of the problem, we are required to find the count (which is generally than 1e6).

    \medskip
    For all of problem in which we are asked max,min or count(below 1e6) we can try to construct the final string itself by following all the question constrain.

    \bigskip
    In current problem, let f(i,j,\_) := length of longest common subsequece when we are allowed to use s1[0..i] and s2[0...j] \textbf{only}.

    lets suppose we are also keeping track of lcs string formed up till now. let string formed till now as[0...idx].

    \medskip
    Now if know the subproblem, then at index idx - \textbf {we need to know as in how many ways we can fill the as[idx]}.

    \medskip
    Ways to fill:
    fill from s1, fill from s2. 
    two case: both same, both different.

   
    \begin{verbatim}
        \\Debug Version code
        int findAns(int i,int j,string as, const string &s1, const string &s2)
        {
            if(i>= s1.size() || j>= s2.size())  //if one is empty => we cannot find anything common now
                return 0; 
                
            //visualize the answer being created
            cc printf("\t:: [%s] (%s,%s)\n",as.c_str(),s1.substr(i).c_str(),s2.substr(j).c_str()); 
            
            int &mans = mem[i][j];
            int val1 = 0,val2=0,val3=0;
            
            if(s1[i] == s2[j])
            {
                val1 = 1 + findAns(i+1,j+1,as+s1[i],s1,s2);
            }
            else
            {
                val2 =  findAns(i+1,j,as+s1[i],s1,s2);
                val3 = findAns(i,j+1,as+s2[j],s1,s2);
            }
            
            //view all path computed at idx
            cc printf("[%s,%d,%d]: (%d,%d,%d)\n",as.c_str(),i,j,val1,val2,val3); 
            
            return mans = max({val1,val2,val3});
        }
    \end{verbatim}

\end{solution}

% TO-DO: add follow up environment here
% Follow-up Questions
Follow-Up Questions:
\begin{enumerate}[(a)]
    \item Find out the lcs string using mem matrix.
    \item  Find out the lcs string using dir matrix + iterative.
    \item Find out the lcs string using dir matrix + recursive.
\end{enumerate}

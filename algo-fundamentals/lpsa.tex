\begin{problem}{Longest Palindromic Subarray}
    Given a string s, find the lenght of longest palindromic subarray in s.

    \footnotetext{LC5}
    
\end{problem}

\begin{solution}

    \begin{guide}
        TO-DO: Remove parenthesis guides hints.
        
        \item First try with generating all substring. Can you do it in $O(n^2)$ ?

        \item
        If you define your problem statement with l..r range. You have the following defination \verb|f(l,r):= | length of longest valid palindromic substring when you are allowed to use \verb|arr[l..r]| \textbf{only}. \\
        The drawback of this is that it convert your complexity to $O(n^2)$

        \item
        \rfl{If you can define what will be the increment during inclusion of idx. Then you can use Inclusion-Exclusion pattern to solve this problem.}
        
        way1: \verb|f(idx,_):=| lenght of longest palindormic subarray when we are allowed to use $arr[0..idx]$ \textbf{only}.

        way2: \verb|f(idx,_):=| length of longest palindromic subarray ending at idx \textbf{that must include idx}. (OR startign at idx and must include idx). 

       \intution{As we know way1 is preferred with subsequence problem and way2 is preferred for subarray problem.}
       Lets try to solve current subarray via way2.

       \item Can you think of a solution using stack?
       
       \item Can you think of a solution that usages expand around center approach?
       
       \item Do you know about Manacher's Algorithm? Which solves this problem in linear time.
    


    \end{guide}
\end{solution}

\begin{solution}[Recursion | must include boundary]

    As this is subarray problem, lets try to solve it as ending pattern.
    let \verb|arr[l,r]| be the lenght of longest subarray that start at l, end at r and \textbf{must include l and r}.

    Now for \verb|f(l,r)| we have two option:\\
    (a) s[l] == s[r] \verb|val1 = 2 + f(l+1,r-1) OR val1 = INF|\\
    (b) s[l] != s[r] \verb|val2 = INF|

    For first case, 2 will be added to subproblem only if substring f(l+1,r-1) is also a palindromic substring. If its not a palindromic then current substring will also not be palindromic.

    \begin{code2}[lpsa recursion | way1]
        int findAnsTwo(int l,int r,const string &s)
        {
            if(l==r) return 1;
            if(l>r) return 0;
            
            int &mans = mem[l][r];
            if(mans != -1) return mans;
            
            int val1=INF;
            int t = findAnsTwo(l+1,r-1,s);
            if(s[l] == s[r])
            {
                if(t != INF)
                    val1 = 2 + t;
            }
            
            return mans = val1;
        }

        //main.cpp
        for(int l=0;l<s.size();l++)
        {
            for(int r=l;r<s.size();r++)
            {
                int mx = findAnsTwo(l,r,s); \\Amortized O(1)
                if(mx >= gmx)
                {
                    gmx = mx;
                    start = l;
                }
            }
        }
         return s.substr(start,gmx);
        
    \end{code2}

    \begin{code2}[lpsa recursion | way2]
        int findAns(int l,int r,const string &s)
        {
            if(l==r) return 1;
            if(l>r) return 0; 
            
            int &mans = mem[l][r];
            if(mans != -1) return mans;
            
            int val1=INF;
            int val2 = INF;
            int val3 = INF;
            
            int t = findAns(l+1,r-1,s);
            if(s[l] == s[r])
            { 
                if(t!=INF)
                    val1 = 2 + t;
            }
            
            {
                 val2 = findAns(l+1,r,s); //just call for subproblem calculation, no use of
                 val3 = findAns(l,r-1,s);
            }
            
            if(val1>=gmx)
            {
                start = l;
                gmx = val1;
            }
          cc printf("[%d,%d]: %d\n",l,r,val1);
            return mans = val1;       
        }
    \end{code2}


\end{solution}

\begin{pratice}

    \begin{enumerate}
        \item LC2472
    \end{enumerate}

\end{pratice}

\begin{praticeHints}
    \item LC2472 palindromice subarray + jump game
\end{praticeHints}
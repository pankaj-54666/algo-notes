\chapter{Fenwick Tree}\label{chp:fenwick_tree}

For \verb|range_update| and \verb|point_update| both in $log(n)$ time, we use fenwick tree.
Fenwick tree has very low code footprint. \footnote{but you must understand the tree strucutre and uppdate sequence to understand how it works.}
\footnote{
\href{https://cs.stackexchange.com/questions/10538/bit-what-is-the-intuition-behind-a-binary-indexed-tree-and-how-was-it-thought-a}{link:Intution Behind Fenwick Tree}}
\footnote{fenwick tree is not suitable for range update}

\intution{Intution: Just like carr[] gives us rangeQuery in O(1), but point update is still O(n). What if we store the array element itself in a balanced binary tree, where each node contains the sum of all its left child nodes.}
With this defination, \verb|q[0,idx]| can be done in log(n) time, and so does point update in log(n). (think it like in both the operations we traversal from leaf node to root)
\newpage
% \begin{fullwidth}

%     \begin{figure}
%         \begin{minipage}[c]{0.5\fullwidthlen}
%             \includegraphics[width=.3\textwidth]{resources/dsa-fenwick-tree-visualization.jpg}
%         \end{minipage}
%         \begin{minipage}[c]{0.3\fullwidthlen}
%           \caption{some caption}
%         \end{minipage}

%       \end{figure}

      
% \end{fullwidth}


    % \begin{figure}
    %     \begin{fullwidth}
    %     \begin{minipage}[c]{0.4\fullwidthlen}
    %         \includegraphics[width=\textwidth]{example-image}
    %     \end{minipage}\hspace{5mm}
    %     \begin{minipage}[c]{0.6\fullwidthlen}
    %            \lipsum[2]
    %            % Hello
    %     \end{minipage}
    % \end{fullwidth}
    % \end{figure}

\begin{lfigure}{resources/dsa-fenwick-tree-visualization.jpg}{0.6}{0.38}
    Any node in the tree has sum of all node in its leftsubtree.
    \vspace{1cm}
    
    \verb|query_parent = idx + [idx & (-idx)]|

    \verb|update_parent = idx - [idx & (-idx)]|

\end{lfigure}

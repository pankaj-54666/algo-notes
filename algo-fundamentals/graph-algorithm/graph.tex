\chapter{Graph Algorithm}\label{ch:graph}

New data-strucuture that will lead us to algorithm on graphs.

The reason why graph algorithm are cool behides having tons and tons of applications is because they force you from a programmers point of view to really distinguish between some intutive idea you have about how to do something and actually making that implementation work.

A algorithm is about two things, clear ideads and getting that intutive idea out of you and describing it, but once you've described it there's  still a way to go and defining that data-structure and make it work.

\section{Graph Representation}

\begin{lfigure}{example-image}{0.5}{0.5}
    \lipsum[3]
\end{lfigure}

\codecaption{Adjancency List}
\begin{lfigure}
    {example-image}{0.5}{0.5}
    
\end{lfigure}

There are general techniquie on how to traverse the graph, namely depth-first-search(dfs) and breadth-first-search(bfs).

DFS and BFS are basic tools that becomes the building block of other fancy graph algorithm.\\
ex: finding  a cycle can be done using a variant of dfs.

A cool example using dfs tool is finding strongely disconnected component in class.

Other graph algorihtms:
\begin{exercise}[Graph Algorithm Summarization]
    \textbf{Famous Graph Algorithms:}
    \begin{compactenum}
        \item  Shortest path
        \item Minium Spanning tree
    \end{compactenum}

    Pratice:
    \begin{compactenum}
        \item Shortest path that visit each vertex at least once. (LC847)
        \item Shortest path the visit each edge at least once.
    \end{compactenum}

    \textbf{Non Famous Algorithms:}
    \begin{compactenum}
        \item Given a graph, can you tell if its a plainer graph or not.
        
        A graph is plainer graph, if you could draw it without crossing any edges. (its importatnt because, some algorithm run faster on plainer graph)
    \end{compactenum}
\end{exercise}

\newpage
\section*{Topological Sorting}
A topological sort of graph make sense only when the graph is directed.

\begin{marginfigure}
    \raggedright
\textbf{algorithm and data-structure tradeoff.}

There is always a trade-off between data strucutre and algorithms.
\vspace*{2mm}

The more fancy things you keep in your data-structure usually the less work your algorithm has to do. Conversly, the less fancy things you keep in your data structure, the more work your algorihtm has to do.
\end{marginfigure}

 For topological sorting, we will first try to solve it without storign anything, then notice that its works a little than we like. Then later when we make our data-strucutre fancier and do a extra work in beginning, then the work we do later is little less.

\begin{marginfigure}
    \vspace{2mm}
    
    \raggedright
    \textbf{
        Never think of a algorithm without a data-structure}.
\end{marginfigure}

\vspace{5cm}
\begin{lfigure}{example-image}{0.5}{0.47}
    You can visualize each node as course of a university, and to complete a course you must complete all course before it.

    Basic Idea:
    \begin{compactenum}
        \item Find all nodes that have no arrow going into them. (indegree)
        \item Delete it. Output it.
        \item Got to step 1 till graph is not empty.
    \end{compactenum}
\end{lfigure}

Will the above algorithm works always? what if you cannot fund a node with 0 indegree but the graph is not dempty.

Above case will only happen if there is cycle is a graph. In fact this is a good way to detect cycle in a direct graph.(though it wont tell you what is the cycle.)

\textbf{Topological sorting only works on DAG.} If the graph has cycle, then Topological orderting has no meaning for that graph.

\medskip
\begin{code3}[Topological Sorting]
    Topological Sorting Code Here
\end{code3}

\section{Minimum Spanning Tree}\label{ch:minimum-spanning-tree}

Minimum spanning tree is a really  baisc graph algorithm, it requires a little more than the brute-force algorithm that we just applied on topolocial sorting.

It serves as a fine example on how data-strucutre and algorithm go altogether.

There are two algorithm to solve this, and they are \textbf{Prim} and \textbf{Kruskal}.

\codecaption{Prims Algorithm | DS: Heap+Graph}
Prim's algorithm create the minimum-spanning-tree in incremental manner. It select a node which has minimum cost and include that in growing tree forest.

\begin{lfigure}{resources/topological.png}{0.3}{0.67}
    Let,\\
    \verb|parent[v]:= parent node of vertex v in minimum spanning tree|
    \verb|cost[v]:= cost to pay if we want to include vertex v in minimum spanning set.|

    Before start, set \verb|parent[u]=-1, cost[u]=INF|

    \begin{compactenum}
        \item set cost[0] = 0
        \item Pick a node from heap that has lowest cost. Let this node be u.
        \item Process this node. Processing means, update the cost[v] and parent[v]. where \verb| u-(w)>v| represent edge from u to v with weight w.
        \item Delete this node and go to step2 again. Repeate this until there is no node in heap. 
    \end{compactenum}

\end{lfigure}

\codecaption{Kruskal Algorithm | DS: union-find+Graph}
\begin{lfigure}{resources/Kruskal.png}{0.3}{0.67}

    In Kruskal's agorithm we select the edge which has minimum weight and include it in tree forest. If the selected edge causes to form a cycle, then we discard such edge and continue processing rest of the edge.

    Kruskal Algorithm usages union-find data-structure to detect if including the edge will result in cycle or not.
\end{lfigure}

Disjoint Union Find with path compression (union-find) is really good example to show  how amortized complexity works.

\section{Graph Traversal}

Both BFS and DFS go through the graph and process each node once. But they do it in different order.
The question is what they are going to do when they go through the node.

Sometimes they can do a very complexity things, which result in different algorithms. All graph algorithm usages bfs/dfs and do different task while going through node, which result in different algorithm.

The data strucutre that go with BFS is queue, the data strucutre that go with DFS is stack.

\marginnote{DFS has much more use than BFS due to its recursive structure.}

\rfl{The key thing about DFS is that, it start going down first, so when it backs its way up, it go so much information, that  the node which has called it can use this data to process node and make decision.}
\section{Shortest Path (TC: greedy)}
% TC is Alogrithm Techniqueue
\marginnote{An anology of -ve weight on graph can be pusing water up the hill.}

\marginnote{Finding shortest from single source to single target, also finds out shortest path to all other nodes. And nobody is able to devise a algorithm that does better for first case only.}[2cm]

There are a lot of versiion of shortest path.
\begin{compactenum}
    \item graph is undirected
    \item graph is directed.
    \item All +ve Edges (aka Dijkastra Algorithm)\\$O(n^2)$ with matrix representatio. and $O(nlog(n))$ with Adjacency List with Heap.
    \item On DAG. (algo. Breadth First Scanning and Bellman Ford) (directed graph but no cycle) (even with -ve edge weight) | $O(e)$
    \item General Case. (-ve weight and can have cycle) (what we are not allowing is that a cycle that has -ve weigh.) |a polynomial time algoritm | $O(n*e)$
    
    Having a -ve weight cycle, means you can have -INF shortest path, which has no pratical meaning.

    \item with -ve weight, and can have -ve weight cycle, but you are not allowed to use such cycle. | a NP Complete Problem.
    
\end{compactenum}

\rule{\linewidth}{0.2em}
\rfl{Shortest path works on Greedy Technique}
\intution{we are gonna keep track of current distance[] and parent[] of the shortest path tree. And we are going to update as we move along. We are gonna in a greedy way by taking local minimuma, if later we found out a new local minimuma for same node, we will scan that node again.  When we are all done with local minima, then we will have our overall best shortest path.}

Order of scan for DAG?\\
If you scan the nodes in topological order, then you are sure that a given node never me scanned twice!

Longest Path vs Shortest Path.\\
Change the edge sign and find shortest path of the DAG.

Longest path in a tree?\\
This is well known problem, known as Diameter of a tree.

\rfl{A tree is a special graph, that does not have a cycle. It means a tree does not have any back edge, and we don't talk about shortest path in tree, we talk about only one path.}

Diameter of a graph?\\
Take any two node of the graph, find the shortet path between them. Then the longest of those will be the diameter.(A graph can have more than one path to other node, while tree has always unique path)
\\This is longest we have ever have to traverl from going one node to other.

Breath First Scanning: Scan node in their BFS order.

\rule{\linewidth}{0.2em}

\begin{exercise}
    \textbf{Theoretical Questions:}
    \begin{enumerate}
        \item What happen if you run Bellman-Form for more than n-1 times?
        \item can Dijkastra work with Directed graph if it has no -ve edge weight?
        \item What is the fastest way to find single source shortest path? In which case it happens.
    \end{enumerate}
\end{exercise}

\begin{exerciseHints}
    \begin{enumerate}
      
    \item If the graph does not have a -ve weight cycle, then running it past n-1 times will not change the dis[] and par[].
    But, if there has a -ve weight cycle, then running it will change the dis[]. (in fact you could use this information to detect -ive weight cycle and even print them.)

    \item Yes,Dijkastra can work with either directed or undirected as long as graph do not have -ve edge weight.
    \item single-source shortest path can be found in $O(e)$ times for a topological graph.
    \end{enumerate}
\end{exerciseHints}
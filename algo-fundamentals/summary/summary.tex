\chapter{Problem Summary}{\label{summary}}

Summarization of various problem and their pragamatic solutions.
\begin{compactenum}
    \item recursion 
    \item backtracking
    \item binary search
    \item divide and conquere
    \item sorting: quick sort, merge sort, bubble sort, radix sort, insertion sort
    \item Tree Algorithms
    \item  Dynamic Programming
    \item Graph Algorithms
    \item Stack,Monotonic Stack
    \item Queue, Monotonic Queue, Sliding Window
    \item He
\end{compactenum}

Topic wise fundamental problem list.

{\smaller[1] 
\begin{exercise}[Most Important Concept]
    \begin{description}
    \item [Dynamic Programming:] Has a lot of variation, but basic is same. i.i to caculate the answer via recursion first, then memoize or further optimize then memoize.
    
        \begin{compactenum}[(a)]
            \item dp without state: subset sum,coin change
            \item dp with l,r pointer: longest palindromic string,longest valid parenthesis
            \item dp with l,r and k: matrix chain multiplication, burst bullon
            
            \item dp with state: stock problem series, paint fence, smallest possible team.
            \item memoization with backtracking
            
            \item palindromic pattern 
            \item valid parenthesis pattern 
        \end{compactenum}

    \item[Tree Algorithms:] Tree has various problem bases on them.
        \begin{compactenum}[(a)]
            \item tree traversal. (inorder,preorder,postorder)
            \item tree traversal level order, zig zag level order, 
            \item LCA, root to leaf path, diagmeter of tree, is tree symmetrical, is tree mirror image
        \end{compactenum}

    \item[Trie Implementation:] Implement trie bases on knowlege of tree.
    
    \item[Graph Algorithms:] Refer Shai Simonson graph copy notes.(its sufficent)
    \begin{compactenum}[a]
        \item DFS
        \item BFS
        \item BackEdge of graph
        \item isCycle
        \item Dijakstra, Bellman Ford
        \item Topological Sort
        \item Articulation Point
        \item Number of disconnected compoonent.
        \item TO-DO: add more
    \end{compactenum}

    \item[Stack:] To check if monotonic can be applied to problem, checkif finding NGL,NGR will help you solve the problem.
        \begin{compactenum}[a]
            \item implement monotonic stack, slope of monotonic stack
            \item minimum element contricution in monotonic increasing stack, max element contricution in monotonic decreasing stack.
            \item NGL,NGR; NSL,NSR calcualtion. 
            
            \item stack supporting basic + getMin: solution using two stack, solution using single variable.
            \item implement stack using queue.
        \end{compactenum}


        \item[Binary Search:] There are a lot of various and this usually can combine with any other algorihtm.
            \begin{compactenum}[a]
                \item Recursive binary search code, iterative binary search code
                \item implement lowerbound, implement upperbound
                \item implement find,lowerbound,infliction point when array is rotated or shifted.
            \end{compactenum}


        \item[Range Bases Query:] Various Data structure.
        \begin{compactenum}
            
            \item Fenwick tree : range query + point update (very small code and quick to implement)
            \item Segment Tree: range query + point update + range update (all in $log(n)$)
            \item sqrt decompoisiton
        \end{compactenum}
        
        \item[Union Find] Disjoin unit set, union by rank, path compression.

        \item[Specilized String Algorithms:] Advanced string search algorithms.
        \begin{compactenum}[(a)]
            \item string hashing. \footnote{https://cp-algorithms.com/string/string-hashing.html}
            \item KMP
            \item Robin-Karp
            \item Suffix Tree (used for Pattern search)
        \end{compactenum}

        \item[bit-manupulation tricks] Tips and Tricks for bit dp / bit problems / bit combined to other problems
        \footnote{https://cp-algorithms.com/algebra/bit-manipulation.html}
        \begin{compactenum}
            \item set bit, pop count
        \end{compactenum}
        

       
        
        
        
\end{description}
\end{exercise}}
\begin{problem}{LC2771 Longest Non-decreasing Subarray From Two Arrays}
    You are give two array A and B of size n. $(n <= 10^5)$. Your task is to return the \textbf{maximum subarray length} in third array C. Where for c[i] in the range [0, n - 1], you can assign either A[i] or B[i].

    \footnotetext{Excellent Problem depicting key concept of recursion + past state in dp + subsequence vs subarray in recursion.}
\end{problem}

\begin{solution}[Recursion | Ways at idx(aka Inclusion-Exclusion)]

    Like all recursio problem subarray problem are not different.
    The only differance they posses from subsequence is that their element must be picked continously. We can include this informatin into pastState variable and code our solution.

    \medskip
    \intution{let $f(idx,pstate):=$ maximum subarray length if we are allowed to use arr[idx...size-1]  ONLY.}

    \medskip
    \rfl{Now as this is a state problem, each of the operation has a constrain applied to them.}

    \medskip
    For this problem, we have following options:
    \begin{enumerate}[(a)]
        \item \icode{int val1}: take from A. \\Restriction: can only be taken if its form a increasing subarray
        \item \icode{int val2}: take from B. \\Restriction: can only be taken if its form a increasing subarray
        \item \icode{int val3}: do not take.\\ 
        Restriction(subarray): for subarray problem, for this to happen we must have not picked any element previously.\\
        Restriction(subsequence):
        For subseqeunce problem, there will be no restriction on this.
    \end{enumerate}
    
    \begin{guide}
        Here is sample hint.

    \end{guide}

    % \newgeometry{left=20mm,top=20mm,right=20mm,bottom=25mm,marginparsep=5mm,marginparwidth=10mm}%
    \begin{lstlisting}[caption=Python example]

/* state=0 := no previous element was selected, state=1:= previous element was selected from A, state=:= previous element was selected from B*/
int findAnsFive(int idx,int pstate,vector<int>& A, vector<int>& B) /* pstate = previous state & NOT the current state, VERY important to make this observation*/
{
    if(idx >= A.size()) return 0;
    
    int &mans = mem[idx][pstate];
    if(mans != -1) return mans;
    
    /* at this time we have three options:
        (a) take from A
        (b) take from B
        (c) do not take (for subsequence this is always valid: but for subarray, this is only valid if we MUST HAVE  NOT PICKED any element previously)
    */
    /* Imp for subset sum, we was having two options: take from array , do not array*/
    
    int val1 = 0,val2=0,val3=0;
    
    int pelement = (pstate == 0) ? -1 :((pstate==1) ? A[idx-1]:B[idx-1]);
    
        if(pstate == 0) /* for subarray we can skip taking current element only if we have not taken any previous element*/
        val3 = findAnsFive(idx+1,0,A,B);
    
    if(pelement <= A[idx])
        val1 = 1 + findAnsFive(idx+1,1,A,B);
    if(pelement <= B[idx])
        val2 = 1 + findAnsFive(idx+1,2,A,B);
    
    
    
    return mans = max({val1,val2,val3});
}
\end{lstlisting}

\begin{code2}[Hello Code]

    /* state=0 := no previous element was selected, state=1:= previous element was selected from A, state=:= previous element was selected from B*/
    int findAnsFive(int idx,int pstate,vector<int>& A, vector<int>& B) /* pstate = previous state & NOT the current state, VERY important to make this observation*/
    {
        if(idx >= A.size()) return 0;
        
        int &mans = mem[idx][pstate];
        if(mans != -1) return mans;
        
        }
\end{code2}
        
\begin{code3}
    /* state=0 := no previous element was selected, state=1:= previous element was selected from A, state=:= previous element was selected from B*/
    int findAnsFive(int idx,int pstate,vector<int>& A, vector<int>& B) /* pstate = previous state & NOT the current state, VERY important to make this observation*/
    {
        if(idx >= A.size()) return 0;
        
        int &mans = mem[idx][pstate];
        if(mans != -1) return mans;
        
        /* at this time we have three options:
            (a) take from A
            (b) take from B
            (c) do not take (for subsequence this is always valid: but for subarray, this is only valid if we MUST HAVE  NOT PICKED any element previously)
        */
        /* Imp for subset sum, we was having two options: take from array , do not array*/
        
        int val1 = 0,val2=0,val3=0;
        
        int pelement = (pstate == 0) ? -1 :((pstate==1) ? A[idx-1]:B[idx-1]);
        
            if(pstate == 0) /* for subarray we can skip taking current element only if we have not taken any previous element*/
            val3 = findAnsFive(idx+1,0,A,B);
        
        if(pelement <= A[idx])
            val1 = 1 + findAnsFive(idx+1,1,A,B);
        if(pelement <= B[idx])
            val2 = 1 + findAnsFive(idx+1,2,A,B);
        
        
        
        return mans = max({val1,val2,val3});
    }
    \end{code3}
\end{solution}
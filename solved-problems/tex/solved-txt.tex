% text solution is added
% NEW PROBLEM
\begin{problem}{LC3026. Maximum Good Subarray Sum}
    Hello THere 2
    \footnotetext{Pratice Link: \href{https://leetcode.com/problems/maximum-good-subarray-sum/}{LC3026}}
\end{problem}

\begin{solution}[greedy,dp,graph]

    \begin{code3}
        Hello There
    \end{code3}
\end{solution}

% NEW PROBLEM
\begin{problem}{LC368 Largest Divisible Subset}
    \footnotetext{Pratice Link: \href{https://leetcode.com/problems/largest-divisible-subset}{LC368}}
\end{problem}

\begin{solution}[Pure LIS (=ending at idx)]

    LIS with parent tracking.
\end{solution}

% NEW PROBLEM
\begin{problem}{LC76 Sort Colors, 3way sort}
    Dutch National Flag Problem
    \footnotetext{Pratice Link: \href{https://leetcode.com/problems/sort-colors/}{LC75}}
\end{problem}

\begin{solution}[hints]
    \begin{code2}
    /* Solutions:

    Approach1: use heap to find kth largest!

    Approach2: Dutch National Flag Problem
        tips for clean code (check sortColors2)
        hint: first move 2 to right, so you need to worry about 0,1 ONLY
    */
    \end{code2}
\end{solution}


% NEW PROBLEM
\begin{problem}{LC2999 Count the Number of Powerful Integers}
    Return the total number of powerful integers in the range [start..finish] also ends with suffix s.
    \footnotetext{Pratice Link: \href{https://leetcode.com/problems/count-the-number-of-powerful-integers/}{LC2999}}
\end{problem}


\begin{solution}[hints]

    \begin{hints}
        count from $0...end - count$ from $0...start-1$
        + recursion 
        + memoization
    \end{hints}
    \begin{code2}
    /*
        Approach1: try to create recursion where the digit range can be picked!

        Appraoch2: other way to solve it, isto split the problem in two part.
            val1 = [0...end] and val2 = [0...start-1] Now, our ans would be val1-val2

            This greately simplifes the problem.

            One for point to notice is that how would you, make sure the the string being created at suffix index does not cross the original numer?!
                ex: end = 1023 and suffix = 24
    */
    \end{code2}
\end{solution}